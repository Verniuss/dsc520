% Options for packages loaded elsewhere
\PassOptionsToPackage{unicode}{hyperref}
\PassOptionsToPackage{hyphens}{url}
%
\documentclass[
]{article}
\usepackage{amsmath,amssymb}
\usepackage{lmodern}
\usepackage{ifxetex,ifluatex}
\ifnum 0\ifxetex 1\fi\ifluatex 1\fi=0 % if pdftex
  \usepackage[T1]{fontenc}
  \usepackage[utf8]{inputenc}
  \usepackage{textcomp} % provide euro and other symbols
\else % if luatex or xetex
  \usepackage{unicode-math}
  \defaultfontfeatures{Scale=MatchLowercase}
  \defaultfontfeatures[\rmfamily]{Ligatures=TeX,Scale=1}
\fi
% Use upquote if available, for straight quotes in verbatim environments
\IfFileExists{upquote.sty}{\usepackage{upquote}}{}
\IfFileExists{microtype.sty}{% use microtype if available
  \usepackage[]{microtype}
  \UseMicrotypeSet[protrusion]{basicmath} % disable protrusion for tt fonts
}{}
\makeatletter
\@ifundefined{KOMAClassName}{% if non-KOMA class
  \IfFileExists{parskip.sty}{%
    \usepackage{parskip}
  }{% else
    \setlength{\parindent}{0pt}
    \setlength{\parskip}{6pt plus 2pt minus 1pt}}
}{% if KOMA class
  \KOMAoptions{parskip=half}}
\makeatother
\usepackage{xcolor}
\IfFileExists{xurl.sty}{\usepackage{xurl}}{} % add URL line breaks if available
\IfFileExists{bookmark.sty}{\usepackage{bookmark}}{\usepackage{hyperref}}
\hypersetup{
  pdftitle={assignment\_06\_RamirezKyle},
  pdfauthor={Kyle Ramirez},
  hidelinks,
  pdfcreator={LaTeX via pandoc}}
\urlstyle{same} % disable monospaced font for URLs
\usepackage[margin=1in]{geometry}
\usepackage{graphicx}
\makeatletter
\def\maxwidth{\ifdim\Gin@nat@width>\linewidth\linewidth\else\Gin@nat@width\fi}
\def\maxheight{\ifdim\Gin@nat@height>\textheight\textheight\else\Gin@nat@height\fi}
\makeatother
% Scale images if necessary, so that they will not overflow the page
% margins by default, and it is still possible to overwrite the defaults
% using explicit options in \includegraphics[width, height, ...]{}
\setkeys{Gin}{width=\maxwidth,height=\maxheight,keepaspectratio}
% Set default figure placement to htbp
\makeatletter
\def\fps@figure{htbp}
\makeatother
\setlength{\emergencystretch}{3em} % prevent overfull lines
\providecommand{\tightlist}{%
  \setlength{\itemsep}{0pt}\setlength{\parskip}{0pt}}
\setcounter{secnumdepth}{-\maxdimen} % remove section numbering
\ifluatex
  \usepackage{selnolig}  % disable illegal ligatures
\fi

\title{assignment\_06\_RamirezKyle}
\author{Kyle Ramirez}
\date{2/13/2022}

\begin{document}
\maketitle

\#**There was an error in trying to make this a pdf. Not sure how to
find it but it would not let me convert this into a pdf.

\hypertarget{set-the-working-directory-to-the-root-of-your-dsc-520-directory}{%
\subsection{Set the working directory to the root of your DSC 520
directory}\label{set-the-working-directory-to-the-root-of-your-dsc-520-directory}}

\#setwd(``/Users/Kyle/Documents/GitHub/KR/Ramirez\_Kyle\_DSC510/dsc520'')

\hypertarget{load-the-datar4dsheights.csv-to}{%
\subsection{\texorpdfstring{Load the \texttt{data/r4ds/heights.csv}
to}{Load the data/r4ds/heights.csv to}}\label{load-the-datar4dsheights.csv-to}}

\#heights\_df \textless- read.csv(``data/r4ds/heights.csv'')

\hypertarget{load-the-ggplot2-library}{%
\subsection{Load the ggplot2 library}\label{load-the-ggplot2-library}}

\#library(ggplot2)

\hypertarget{fit-a-linear-model-using-the-age-variable-as-the-predictor-and-earn-as-the-outcome}{%
\subsection{\texorpdfstring{Fit a linear model using the \texttt{age}
variable as the predictor and \texttt{earn} as the
outcome}{Fit a linear model using the age variable as the predictor and earn as the outcome}}\label{fit-a-linear-model-using-the-age-variable-as-the-predictor-and-earn-as-the-outcome}}

\#age\_lm \textless- lm(age \textasciitilde{} earn, data = heights\_df)

\hypertarget{view-the-summary-of-your-model-using-summary}{%
\subsection{\texorpdfstring{View the summary of your model using
\texttt{summary()}}{View the summary of your model using summary()}}\label{view-the-summary-of-your-model-using-summary}}

\#summary(age\_lm)

\hypertarget{creating-predictions-using-predict}{%
\subsection{\texorpdfstring{Creating predictions using
\texttt{predict()}}{Creating predictions using predict()}}\label{creating-predictions-using-predict}}

\#age\_predict\_df \textless- data.frame(earn = predict(50000, 70000),
age=45)

\hypertarget{plot-the-predictions-against-the-original-data}{%
\subsection{Plot the predictions against the original
data}\label{plot-the-predictions-against-the-original-data}}

\#ggplot(data = age\_predict\_df, aes(y = earn, x = age)) + \#
geom\_point(color=`blue') + \# geom\_line(color=`red',data =
age\_predict\_df, aes(y=earn, x=age))

\#mean\_earn \textless-
mean(heights\_df\(earn) ## Corrected Sum of Squares Total #sst <- sum((mean_earn - heights_df\)earn)\^{}2)
\#\# Corrected Sum of Squares for Model \#ssm \textless- sum((mean\_earn
- age\_predict\_df\(earn)^2) ## Residuals #residuals <- heights_df\)earn
- age\_predict\_df\$earn \#\# Sum of Squares for Error \#sse \textless-
sum(residuals\^{}2) \#\# R Squared R\^{}2 = SSM\SST \#r\_squared
\textless- ssm/sst

\hypertarget{number-of-observations}{%
\subsection{Number of observations}\label{number-of-observations}}

\#n \textless- 3 \#\# Number of regression parameters \#p \textless- 2
\#\# Corrected Degrees of Freedom for Model (p-1) \#dfm \textless- p - 1
\#\# Degrees of Freedom for Error (n-p) \#dfe \textless- n - p \#\#
Corrected Degrees of Freedom Total: DFT = n - 1 \#dft \textless- n - 1

\hypertarget{mean-of-squares-for-model-msm-ssm-dfm}{%
\subsection{Mean of Squares for Model: MSM = SSM /
DFM}\label{mean-of-squares-for-model-msm-ssm-dfm}}

\#msm \textless- ssm / dfm \#\# Mean of Squares for Error: MSE = SSE /
DFE \#mse \textless- sse / dfe \#\# Mean of Squares Total: MST = SST /
DFT \#mst \textless- sst / dft \#\# F Statistic F = MSM/MSE \#f\_score
\textless- msm / mse

\hypertarget{adjusted-r-squared-r2-1---1---r2n---1-n---p}{%
\subsection{Adjusted R Squared R2 = 1 - (1 - R2)(n - 1) / (n -
p)}\label{adjusted-r-squared-r2-1---1---r2n---1-n---p}}

\#adjusted\_r\_squared \textless- r\_squared = 1 - (1 - r\_squared)(n -
1) / (n - p)

\hypertarget{calculate-the-p-value-from-the-f-distribution}{%
\subsection{Calculate the p-value from the F
distribution}\label{calculate-the-p-value-from-the-f-distribution}}

\#p\_value \textless- pf(f\_score, dfm, dft, lower.tail=F)

\end{document}
